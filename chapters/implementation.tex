%************************************************
\chapter{Implementation}\label{ch:implementation}
%************************************************
In order to validate the metric,
an application that appealed both to developers as well as employers
needed to be built. It should motivate developers enough to grant it access to their
public repositories, and convince recruiters to use it. The special means that
the development of the application required will be described in this chapter.

\section{Architecture}
Web applications are state-of-the art as they require little servicing
and no complex versioning. The use-case of many users taking
advantage of a common dataset was perfect for it.
We decided to implement the serverside logic in NodeJS, which is a
JavaScript runtime. The client-side runs in the browser and makes
use of standard web technologies like CSS3, HTML5 and Javascript.
The analysis results are saved in a SQLITE3 database.
\newline

The serverside part needs to master two tasks: serving data to clients
as well as analyzing repositories from developers.
These are obviously two very different concerns.
In order not to stall one of the tasks while executing the other,
we decided to have two dedicated processes running alongside each other.
One of these will serve the data and perform queries on the database,
while the other will keep the local copies of repositories up to date and run
the metrics on them. We will call the first the \textit{hirebot} and the
latter the \textit{analyzer}.

\subsection{Database schema}
The database schema of hirebot is not very complex.
Users are central to the application and every other type of data
is associated with one, as seen in figure \ref{fig:schema}.

\subsection{Hirebot}
The component that plays the webserver role is called \textit{hirebot}.
It attends the usual webserver tasks like formatting templates and querying
the database for answering HTTP requests to an endpoint.
\newline

Additionally it implements the foundation for future data analysis.
It allows developers to register with GitHub to grant the application
access to their emails and their repositories. Hirebot then notifies
the analyzer of this additional datasource. The repositories are then cloned
and analyzed. This process is also depicted in figure \ref{fig:regprocess}.

%\begin{figure}[ht]
%  \centering
%  \includegraphics[width=35em]{gfx/registersequence.png}
%  \caption{The interplay of the three processes after the OAuth 2.0 dance has been completed.}
%  \label{fig:regprocess}
%\end{figure}

\subsection{Analyzer}
The analyzer component of the application handles two slightly different
tasks. First, it has to schedule an initial clone of all repositories
a user owns upon his registration. Then, it has has to keep those local
copies up to date and add new repositories, should there be any.


These are certainly no impossible tasks, but \textit{nodegit}, the library
that was used for accessing the git repository data, was
leaking memory strongly and to counter this, a sub-process structure was
implemented. The analyzer itself only executes management logic,
schedules the analysis, and communicates with the hirebot process.
For executing the analysis, it forks an \textit{analyzer-worker} for each
single repository. The worker clones or updates that repository,
runs the analysis, saves the results into the database and terminates
upon completion. The forks are done sequentially to avoid putting
too much load on the system at once.

\subsection{Algorithm implementation}
The metric described in section \ref{sec:technicalfit} is fully implemented
in the analyzer component. Its implementation works on a per-repository
basis and can be written down like this:

\begin{lstlisting}[frame=false]
for c:=mostRecentlyAnalyzedCommit to lastCommit do
  for d:= c.firstDiff to c.lastDiff do
    var language = determineFileLanguage(d.file)
    var lineCount = d.lineCount
    saveExperience(language, lineCount)
\end{lstlisting}

Additionally, a special JavaScript analyzer was built into the algorithm.
It adds McCabe and Halstead complexity metric information about single
commits. Thus, the final implementation amounts to:

\begin{minipage}{\linewidth}
\begin{lstlisting}[frame=false]
for c:=mostRecentlyAnalyzedCommit to lastCommit do
  for d:= c.firstDiff to c.lastDiff do
    var language = determineFileLanguage(d.file)
    var lineCount = d.lineCount
    saveExperience(language, lineCount)

    if language == 'JavaScript'
      var fileContentsBeforeCommit = d.file.contents
      var fileContentsAfterCommit  = d.parent.file.content

      var mcCabeComplexityBefore = getMcCabeComplexity(fileContentsBeforeCommit)
      var mcCabeComplexityAfter = getMcCabeComplexity(fileContentsAfterCommit)
      var halsteadComplexityBefore = getHalsteadComplexity(fileContentsBeforeCommit)
      var halsteadComplexityAfter = getHalsteadComplexity(fileContentsAfterCommit)

      saveJavaScriptMeasures(mcCabeComplexityBefore,mcCabeComplexityAfter,halsteadComplexityBefore,halsteadComplexityAfter)
\end{lstlisting}
\end{minipage}

\begin{figure}
  \centering
  \includegraphics[width=35em]{gfx/schema.png}
  \caption{The hirebot database schema. Only attributes that are mined from the code are listed.}
  \label{fig:schema}
\end{figure}

\section{Interface}
\subsection{Candidate view}
A candidate can view his general personal statistics and the JavaScript statistics
mentioned above. This provides him with insight about what job offers he might
receive and what to improve about his coding style (see figure \ref{fig:candidateview})

\begin{figure}
  \includegraphics[width=30em]{gfx/candidateview.png}
  \caption{The candidate view of Hirebot}
  \label{fig:candidateview}
\end{figure}

\subsection{Employer view}
In a small small search form, a recruiter can state his requirements concerning
programming language knowledge and for how long this skill should be in training (see figure \ref{fig:employerview}).

\begin{figure}
  \includegraphics[width=30em]{gfx/employerview.png}
  \caption{The employer search form of Hirebot}
  \label{fig:employerview}
\end{figure}
