\chapter{Taking measurements}
This chapter is going to deal with finding an answer
to the question asked in section \ref{subsec:dev-skill-measurement},
how to measure developer skill requirements from job advertisements.

\section{Measuring job skill requirements}
%As stated in the abstract, technical interviews require more
%than the usual interview effort. In addition to personality
%tests that confirm that the candidate fits into the company
%culture, technical knowledge needs to be thoroughly checked.

%This is done in small programming exercises (either on-site or
%outsourced), examplary projects or by directly asking technically
%related questions.

\subsection{Job posting aspects}
By manually analyzing job offers from the Github Jobs
site\footnote{https://jobs.github.com}, it became clear that most
job offers consisted out of three parts:

\begin{itemize}
\item A description of the position environment
\item A detailed description of technical skill requirements
\item A "wishlist" about the employee character traits
\end{itemize}

Most startup job postings put the emphasis on candidate personality and
willingness to learn, as tasks are not yet so clearly defined at this stage
of company life. For this reason more postings from larger, established
companies were taken into account.


Of these job postings, for example GitHub and Apple had very
specific, measureable technical requirements:
\newline

Apple wants candidates for a data engineer position to
\begin{itemize}
    \item have 3+ years experience with SQL
    \item have 3+ years experience with NoSQL
    \item know Hadoop
\end{itemize}

GitHub wants candidates to have experience
\begin{itemize}
    \item with web application backend, 3+ years
    \item with SQL
    \item with Ruby, JavaScript, ElasticSearch optionally
    \item with AWS or similar computing solutions optionally
\end{itemize}

Optional qualifications were always mentioned in conjunction with the
word "bonus". Presumably, candidates who receive bonuses are more
likely to get hired.


\section{Determining developer skill}