%************************************************
\chapter{Conclusion}\label{ch:conclusion}
%************************************************
\section{Future work}\label{sec:future-work}
The Hirebot platform is a very basic prototype and extensible in many directions. In the following, improvements that might improve usability and ranking are described.

An account on GitHub is certainly very promising, but it is not the only place where one can publish code. There are similar services like Gitorious, Bitbucket or GitLab and of course self-hosted git-repositories. If one has made significant contributions to these codebases, these will not show up in the final statistics, as outlined in section \ref{sec:threatstovalidity} and criticized in section \ref{sec:key-findings}. As such, it should at least be possible to manually add other repositories by hand, or connect other platforms in a process similar to that used with GitHub.

The connection with other platforms leads to another possible improvement that might cancel out false recommendations, also a critic picked up during the interviews. Instead of relying on a single source, it would make sense to pull in information from others to improve the basis for decision. There are numerous job profile sites like XING\footnote{\url{http://xing.de} | checked June 10, 2015} or LinkedIn\footnote{\url{http://linkedin.com} | checked June 10, 2015} where users have published their complete curriculum vitae. A profile here can hint at the suitability of a candidate for a position. For example a user that is found to be suitable for a data engineer position might not want to leave his job at a technology consultancy firm - he is just developing in his own interest. While it is possible to indicate availability by using the hirable flag (see section \ref{sec:data-source}), very many GitHub users do not use this feature, making external data necessary. Other interesting sources include Openhub\footnote{\url{http://openhub.net} | checked June 10, 2015} and StackOverflow\footnote{\url{http://stackoverflow.com} | checked June 10, 2015}. By using the techniques for identifying developers across StackOverflow and GitHub\cite{vfs:2012}, it could be possible to enrich user profiles automatically to better identify personal topic preferences. With this information at hand, building something similar to a Klout score\footnote{\url{http://mashable.com/2011/06/29/work-for-pie} | checked June 10, 2015}. is possible

The reliable identification of candidates across several sources also rids the system of the necessity of registrations. Developers can easily be \textit{found} by crawling the GitHub API, thus increasing the user base for recommendations.


Lots of job advertisements also put emphasis on certain technologies. For instance it does not suffice to know JavaScript - specifically AngularJS (a client-side JavaScript framework) needs to be well known for at least three years. With the current state of development this kind of differentiation is not possible. No distinctions between frontend or backend code are made, nor are specific technologies considered. This would require technology-specific context and a complex distinction logic as the lines between frontend and backend code begin to blur. For example, Node.js backend-code can easily be run in any web browser using browserify\footnote{\url{http://browserify.org} | checked June 10, 2015}.
\newline

To improve the metric, it makes sense to analyze code sections before and after the commit for changes in complexity and other quality measures. As shown by Giese \cite{pg:2014} and in section \ref{sec:metric-implementation}, this is just another step in the analysis process. Now, if a developer strives to reduce complexity with his commits, this can be rewarded with a boost in the overall ranking. Likewise, complexity-increasing commits should be punished. Once more languages than JavaScript are covered, this seems a worthwhile investment of development effort: employers get into contact with highly skilled developers while students who are good at coding can move up the ranks.

\section{Conclusion}
In this thesis we have constructed a metric as an approach to measuring developer skill. Even though Yashamita et al. claim that software metrics alone can not be used to assess source code\cite{mlya:2012}, it is possible to use these metrics to match code authors to job opening skill requirements by comparing provided and demanded skills. At the heart of research lies Hirebot, a platform where developers can sign up with their GitHub account. Upon registration their public repositories are cloned and data to construct the metric on is gathered. Developers can view their own statistics to get a feeling for what programming languages they have provided data and recruiters can easily enter programming language requirements and receive candidate suggestions. Hirebot was built specifically to verify our metric with real data. As demonstrated in the previous chapter, the results are acceptably good for the amount of data gathered, and we are confident to have found a good solution to the problem.

Having demonstrated the feasibility of an algorithmic matching approach of developers to technical job openings, we believe that it is only a question of time and effort until anyone can be matched on to any job opening automatically.
%The rankings will take into account more and more data, perhaps even checking for a cultural or personality fit.