%************************************************
\chapter{verification}\label{ch:verification}
%************************************************
The whole purpose of the implementation was acquiring real data for finding an answer to research question 2 from section \ref{subsec:measurement-quality}. We did so by conducting a series of short interviews. Most users who registered with the application during the testing phase were HPI students, so finding partners for interviews did not prove a great difficulty.

\section{Interviews}\label{sec:interviews}
The objective of each interview was to verify how the interviewee personally viewed his suitability for a certain position. Finding out about personal interest in the position is not a goal. Interviewees should try to apply a metric similar to ours, i.e. based on their technical abilities. Afterwards their answers were compared to the recommendations made by the algorithm.

Three candidates were chosen by looking at the ranking results generated by Hirebot. The job openings were chosen in such a way that they required different skills, and the ranking produced different results. The candidates were interviewed individually. They were given the technical requirement summaries of 4 job openings on small cards and should make statements about their suitability for them. Afterwards, they were shown the real job offering, complete with the introductory text. Any deviations from their previous statements were noted. Afterwards, the suitabilities expressed by the candidates were compared to the suitabilities determined with our metric.

In the following, job opening shall be abbreviated with JO.

\subsection{Reference job openings}
We used four job openings from four very different institutions, two small startups, a university and a huge established corporation. All of them can be viewed in their entireity in the Appendix \ref{chap:appendix}. They were summarized as follows:

\begin{enumerate}
\item Two years of Python, Knowledge of Java/C++ and SQL.
\item Three years of Ruby and 3 years of JavaScript
\item Five years of C++
\item One year of HTML, JavaScript and CSS
\end{enumerate}

\subsection{Interview 1}
``Not a perfect match, but nonetheless interesting because of Java and C++ or Ruby" were the words spoken in reaction to seeing the summaries to JO1 and JO2. JO3 was also not called a good match, because not the full five years worth of experience with C++ could be provided. JO4 was titled a ``perfect fit".\\
These opinions also did not change greatly upon viewing the whole job opening descriptions, except for JO3. Upon reading the whole description, he strongly argumented against this position for he knew nothing of astrology and did not want to work in the academic field. He mentioned that the programming language requirements were generally helpful at a glimpse but he would have wished for technology requirements and a short description of the work environment.

\subsection{Interview 2}
JO1 was declared a bad fit, because he only knew some of the programming languages required, specifically not Python. Upon reading the whole job offer however, he announced that he could do it because he could learn Python quickly. This no-then-yes-pattern was repeated with JO2. At first the job was called a bad fit because of the required proficiency in Ruby. After reading the whole posting, where the necessity of this was verbally reduced and some used technologies like AngularJS were mentioned, he repositioned himself with a clear approval. JO3 was rejected both times because of lack of skill, and JO4 was accepted both times because the required skills could easily be provided.

\subsection{Interview 3}
This candidate was positive about almost every job offer. JO1 was called a good match. JO2 was turned down, because his experience in both Ruby and JavaScript was not very advanced. JO3 und JO4 were partial fits, because there he did not have a lot to show for C++ or web frontend development. This attitude did not change after going through the whole text of each offering. He also mentioned that while Node.js was technically JavaScript, he would have liked the summary to mention it, because he saw a strong distinction between JavaScript used in frontend and backend.

\section{Key findings}\label{sec:key-findings}
All candidates stated that they were equally interested in technologies as well as programming language requirements. Also, interview partner 1 stated to have capabilities in C++. These were not tracked because he did not have any repositories on GitHub which contained source code written in that language. Another caveat that was described as unwanted was the suggestion of a job in academia. Both of the issues mentioned can be addressed via means that are described in more detail in section \ref{sec:future-work}.

\section{Candidate fit}
As outlined in section \ref{sec:interviews}, the interviewees were chosen with their ranking known. A comparison of where the ranking displayed a total fit and where the candidates saw themselves fit can be seen in table~\ref{table:rank-comparison}.

\begin{table}
\centering
\begin{tabular}{c|cccc}
Candidate \# & JO1 & JO2 & JO3 & JO4\\
\hline
           1 & & $\frac{1}{2}$C & & C|$\frac{2}{3}$R\\
           2 & & $\frac{1}{2}$C|R & & C|R\\
           3 & C|R & $\frac{1}{2}C$|$\frac{1}{2}R$ & C & \\
\end{tabular}
\caption{A comparison of ranking results and the candidates own opinions, because candidates should know their own suitability for a job best. An R marks ranking fit, and a prefixed fraction marks partial fit. The same goes for C, demarking candidate opinion.}
\label{table:rank-comparison}
\end{table}

The ranking matched candidate opinion almost every time. Given the fact that the candidates had very rich GitHub profiles sporting an average of 23 repositories, our metric seems to work reasonably well if enough data is supplied. A minimum repository count of 10 seems to work out fine\footnote{This value has been determined in a series of small experiments and will probably be subject to change.}. Possibilities on how to improve the ranking are outlined in the following and last section. Nonetheless we conclude that the answer to the second research question is a clear yes. Our metric is a good step into the direction of candidate identification.