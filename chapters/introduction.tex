%************************************************
\chapter{Introduction}\label{ch:introduction}
%************************************************
The war for talent is on. Coined at the beginning of the new
millenium, the term describes the increasing competition between companies
for skilled employees\footnote{\url{https://en.wikipedia.org/wiki/The_war_for_talent} | checked June 13, 2015}.

It is on, and it has never been more intense for technical experts\footnote{\url{http://tcrn.ch/1SXBKJl} | checked June 12, 2015}.
A perk that one company offers with one job is countered with another perk by
another company, and vice versa. Startup companies are being bought,
just for the acquirer to access much needed talent.
These so called "aquihires" often lead to an integration of the team into
the company and a cancellation of their old product\footnote{\url{http://bit.ly/1FuPbM7} | checked June 13, 2015}\footnote{\url{https://en.wikipedia.org/wiki/Acqui-hiring} | checked June 13, 2015}.
It even goes so far that companies bid for talent on online marketplaces, like
Hired\footnote{\url{https://hired.com} | checked June 19, 2015} or StartupCV\footnote{\url{https://www.startupcvs.com/} | checked June 19, 2015}.
\newline

Looking at programmers, a candidate for an opening is often put into
test projects or given small programming tasks to solve.
These should prove programming capability, and given a successful completion
they are hired\footnote{\url{http://bit.ly/1LDNtvn} | checked June 13, 2015}.
Both for developers and recruiters, this process feels slow and tedious,
often resulting in frustration for the two. There should be ways of
identifying such capabilities beforehand, so that such timely and financially
costly checks need not be as intensive.

\section{Contribution}
% See phil giesess thesis for a very good introduction on what he has built
% take job advertisements and find developers for them
% goal is a system similar to N2O but for recruiters and not for salespeople
%What has been built in this thesis and what has been achieved?
%GitHub provides users with the possibility of
%publicly hosting source code as well as collaborating with other users on a
%project-scale or even an organization-scale. The source code itself as well
%as the whole git history is publicly available and provides a solid base
%for analysis. Because of its widespread adoption, this platform has been chosen
%for the analysis. Of course, the principle of analyzing source code and matching
%it to job postings is very general and could be applied
%to any kind of repository.
In this thesis we will present a metric with which developers can be matched
against job requirements and be ranked by suitability for it.
The metric will build on top of data gathered from analyzed open source repositories.
In order to verify it, we also present a platform on which recruiters can easily
gain access to these software engineering talents. The recruiters can specify
what kind of position they want to find candidates for, and will receive
suggestions from the userbase that has registered with the application.

Github will be the only source of code in our case, but it has proven
to be sufficiently large and well-frequented for our use case.
The analysis metric has been constructed specifically for our use case,
because to the best of our knowledge, no pre-existing method was suitable.

\section{Research Questions} \label{sec:research-questions}
To guide our research, we formulated two research questions:

\subsection{How can developer skills be measured?}\label{subsec:dev-skill-measurement}
We argue that a measurement metric that is common to most job opening
descriptions must be found. This reduces the need for a lot of customization
on the ad and allows analysis of pre-existing data.
\newline

All developers will be matched against the same job advertisements and thus
they all need to be measured using the same method. The metric that
has been derived from the advertisements will need to be measured
on each developer - provided he or she can provide the necessary data.
Thus the practicability of a metric depends largely on the availability
of measureable data.

\subsection{Can qualified candidates be found automatically?}\label{subsec:measurement-quality}
The metric allows developers to be ranked by suitability for the position.
The qualities of this ranking are ultimately determined by whether
or not the right candidates land at the top of it. This needs to be
verified with real world job candidates and the job opening descriptions
that they applied for

\section{Related Work}
Marlow et al.\cite{md:2013} have conducted an extensive study with about
200 participants, and found out that employers value
open source coder profiles greatly (page 4). The authors prove that developers are conscious of the fact
that their source code is public and thus try to deliver it with
great care. This is relevant for us for two reasons:
First, it proves that contribution to open source has a relevancy
in hiring and is awarded with recognition. And second, Marlows and Dabbishs
research hints at the fact that developers are trying hard to deliver very
good code, because they are aware of its publicity. Because we will use
exactly this code in our measurements, it is highly relevant that it be as good as its author can possibly write it.
Altogether their research has proven that both public sourcecode repositories
are relevant for making recruitment decisions and that open source developers
are aware of this fact and try to deliver their best.
\newline

The suitability of source code analysis for the identification of personal
skillsets has been demonstrated in the master thesis of Philipp Giese\cite{pg:2014}.
He built a framework called \textit{Analyzr}\footnote{\url{https://github.com/frontendphil/analyzr}}
that analyzes the complete history from a git or SVN source code repository.
Based on the code, different metrics are calculated, from which statements about
code complexity and refactoring needs can be derived. It is also possible
to make statements about developer capabilites and areas of expertise.
Thus, given a little context about the analyzed source code, it would possible
to say that \glqq Developer X is mainly a backend developer and the single maintainer of module Y\grqq.

Philipp's findings are supported by that of
Aftab Iqbal and Michael Hausenblas, who have analyzed different
source code repositories for common developers\cite{ih:2012}.
They have done so using relatively little local code analysis and relied
on on-the-fly analysis from retrieved RDF data. Their work
has made it possible to find out whether different open source projects shared developers
and contained useful methods of retrieving authors and eleminating duplicates
caused by different e-mail addresses, for example.
One of their primary sources of code was \textit{sourceforge}, a site that was
popular for open source code hosting before GitHub took its place.
Now, in 2015, sourceforge has more or less disappeared from the map of open
source developers and does not host a lot of projects of relevance anymore.

Vasilescu, Filkov and Serebrenik of TU Eindhoven have taken it further
and integrated developer-related information across several code forges
with crowdsourcing platforms such as StackOverflow\cite{vfs:2013}.
Their findings allow identifying active contributors on StackOverflow
as well as active developers on GitHub.
\newline

A japanese research group at the Kobe University in Tokio, seems
to display great interest in software analytics\cite{mn:2011}. Shinsuke Matsumoto and Masahide Nakamura
have built a web service that allowed the analysis of any SVN or git source code repository.
They were specifically interested in creating a platform for other researchers to
contribute their source code metrics to. The metrics are programming-language agnostic
and very general. Unfortunately the code was badly documented and if there was documentation
to be found, it was written in japanese. Matsumotos and Nakamuras paper
went into deep technical detail of the inner workings of their platform.
This knowledge was particularly helpful in the forming of an architecture for ours.

Matsumoto, Nakamura and Giese also hinted at several code metrics of historical
importance. They made use of McCabes Cyclomatic Complexity metric\cite{mc:1976}
and the Halstead complexity measures\cite{h:1977}.
The first provides a way for measuring computational complexity, which is
something that might hint at a clean design and a performant implementation.
Quite similarly, the latter focuses itself on static code as well, but with the
distinction of providing several metrics, like the presumable number of bugs and
a code difficulty indicator, all based on four small code-related numbers.
\newline

Algorithmically indentifying talent is something that has been tried by many
companies with recruiting at the core of their business model.
For LinkedIn, the tools for recruiters that are fueled by this research are
their main source of income\footnote{\url{http://tnw.co/1g20G4m} | checked June 29, 2015}.
A startup called \textit{Work For Pie}\footnote{\url{http://www.workforpie.com/} | checked June 27, 2015}
built a platform similar to ours and got acquired by the hiring platform GroupTalent.
For this reason it was very hard to find information on the inner workings of the
job recommendation system in production, which is why there is no comparison with their
recommendation techniques in this thesis.

\section{Project background}
Money makes the world go round. That is why the sales department
of any company is vital to its success. Ben Horowitz correctly
stresses this importance: "The sales organization is the face of the company to the outside world"\cite{bh:2014}

As social media platforms like facebook, Twitter, LinkedIn and Xing
have grown successful, they have become something that this face needs to be aware of.
On these platforms, millions of entries are made each day.
500 million tweets were sent on a single day on Twitter in 2013 alone.
All platforms combined result in a huge marketing reach,
making social media channels a huge possibility for sales.
Customer interactions are now possible on a very personal level,
something that seemed nearly impossible a long time ago.
Customer appraisal and shitstorms lie very much near each other.
\newline

In all of this beauty of customer service there is a caveat:
Trying to find the \textit{right} information inside this
flood of data, which is often referred to as noise, is a tiresome
process that needs to be gone through everyday and often by hand.
Tiresome during freetime and even more stressful if it has to be done on the job.
Finding pieces of information which might hint at an interest
in a product are the things are the ones sought after daily by sales representatives.
Such a finding often leads to an attempt at contacting the author of the
entry in question. The information that hints at a consumer interest is commonly
referred to as Opportunity.

Thus, converting "Noise to Opportunity", is a very common struggle
in sales departments. The principle is usually illustrated with a funnel,
as shown and explained in figure \ref{fig:sales-funnel}.

\marginpar{Converting noise to opportunity is the very tiring, repetitive task
of finding useful information in a heap of thematically
but not practically relevant data}

\begin{figure}[bth]
    %\includegraphics[width=90\linewidth]{gfx/sales-funnel}
    \caption{The sales funnel for reducing noise to opportunity. Too much
    information with a few useful bits in it goes in at the top and gets
    reduced to promising pieces that might be converted to a sales opportunity}
    \label{fig:sales-funnel}
\end{figure}

The bachelor project M1 at the chair of Internet Technologies and Systems
at the Hasso-Plattner-Institute tackled this problem by applying demand-based filtering to it.
With product descriptions of the products in mind that the department wants to sell,
the relevant pieces of information are filtered out and delivered to each employee -
ranked by relevancy. Based on different sales territories,
these pieces can also be routed to different individuals.
\newline

\marginpar{The bachelor project is a two-semester project at HPI,
where 2-7 students work together on a large software system,
often for an external partner. In the process, expertise
for the bachelor thesis is gathered. }

The general principle of reducing information load by applying
demand-based-filtering can also be tailored to fit recruiting processes.
Recruiting may even be mapped roughly on to the problem of selling products.
The \textit{products} are the job vacancies in this case, whereas the
\textit{customers} are the potential employees that need to be found.\\
As shown before, there is a heavy competition among recruiters for excellent
technical candidates because of their scarcity and it seems a resonable thought
to gain a competitive edge by pre-analyzing technical candidates
before considering them for an interview. Also, it makes sense to apply
some sort of metric for measuring skill, because
"You can't manage what you can't measure"\cite{tdm:1986}.  People-based metrics
need to be seen skeptically, but hiring is such an important decision that it
should be manifested with numbers, because it can severely affect
company success\cite{hk:1998}.