%************************************************
\chapter{Introduction}\label{ch:introduction}
%************************************************
Money makes the world go round. That is why the sales department
of any company is vital to its success. Ben Horowitz correctly
stresses this importance: "The sales organization is the face of the company to the outside world" \cite{bh:2014}
As social media platforms like facebook, Twitter, LinkedIn and Xing
have grown successful, they have also become relevant for
sales success. On these platforms, millions of entries are made each day,
and have amounted to a sheer unmanageable amount of input for a human.
500 million tweets were sent on a single day on Twitter in 2013 alone.
All platforms combined results in a huge marketing reach.
\newline

But there is a caveat to this: Trying to find relevant information inside this
flood of data, which is often referred to as \textit{noise}, is a tiresome
process that needs to be gone through everyday and often by hand.
Tiresome during freetime and even more stressful if it has to be done on the job.
Pieces of information which might hint at an interest in a product are the things
you are things sought after by sales department employees.
Such a finding often leads to an attempt at contacting the author of that
piece information. Such a piece of sales-relevant information is commonly
referred to as \textit{Opportunity}.
\newline

Thus, converting "Noise to Opportunity", is a very common struggle
in sales departments. The principle is usually illustrated with a funnel,
as shown and explained in figure \ref{fig:sales-funnel}.

\marginpar{Converting noise to opportunity is the very tiring, repetitive task
of finding useful information in a heap of thematically
but not practically relevant data}

\begin{figure}[bth]
    %\includegraphics[width=90\linewidth]{gfx/sales-funnel}
    \caption{The sales funnel for reducing noise to opportunity. Too much
    information with a few useful bits in it goes in at the top and gets
    reduced to promising pieces that might be converted to a sales opportunity}
    \label{fig:sales-funnel}
\end{figure}

The bachelor project M1 at the chair of Internet Technologies and Systems
at the Hasso-Plattner-Institute tackled this problem by applying demand-based filtering to it.
With product descriptions of the products in mind that the department wants to sell,
the relevant pieces of information are filtered out and delivered to each employee -
ranked by relevancy. Based on different sales territories,
these pieces can also be routed to different individuals.
\newline

\marginpar{The bachelor project is a two-semester project at HPI,
where 2-7 students work together on a large software system,
often for an external partner. In the process, expertise
for the bachelor thesis is gathered. }

The general principle of reducing information load by applying
demand-based-filtering can also be tailored to fit recruiting processes.
Recruiting may even be mapped roughly on to the problem of selling products.
The \textit{products} are the job vacancies in this case, whereas the
\textit{customers} are the potential employees that need to be found.\\
It is no secret that there is a heavy competition among recruiters for excellent
technical candidates because of their scarcity and it seems a resonable thought
to gain a competitive edge by pre-analyzing technical candidates
before considering them for an interview. Also, it makes sense to apply
some sort of metric for measuring skill.
"You can't manage what you can't measure"\cite{tdm:1986}, although people-based metrics
need to be seen skeptically. Nonetheless, hiring is an important decision and
should be manifested with numbers, because it can severely affect
company success\cite{hk:1998}.
\newline

As previous studies have shown, the awareness of their source
code being public increases developer determination
to produce top quality source code\cite{md:2013}.
Public source code repositories have grown in importance for recruting
staff, which shows that there is a practical need for determining
employee fit to certain positions before making first contact.

\section{Contribution}
% See phil giesess thesis for a very good introduction on what he has built
% take job advertisements and find developers for them
% goal is a system similar to N2O but for recruiters and not for salespeople
%What has been built in this thesis and what has been achieved?
%GitHub provides users with the possibility of
%publicly hosting source code as well as collaborating with other users on a
%project-scale or even an organization-scale. The source code itself as well
%as the whole git history is publicly available and provides a solid base
%for analysis. Because of its widespread adoption, this platform has been chosen
%for the analysis. Of course, the principle of analyzing source code and matching
%it to job postings is very general and could be applied
%to any kind of repository.

\section{Research Questions} \label{sec:research-questions}

To guide our research, we formulated the following research questions.

\subsection{How can developer skills be measured?}\label{subsec:dev-skill-measurement}
We argue that a measurement metric that is common to most job opening
descriptions must be found. This reduces the need for a lot of customization
on the ad and allows analysis of pre-existing data.
\newline

All developers will be matched against the same job advertisements and thus
they all need to be measured using the same method. The metric that
has been derived from the advertisements will need to be measured
on each developer - provided he or she can provide the necessary data.
Thus the practicability of a metric depends largely on the availability
of measureable data.

\subsection{Can qualified candidates be found automatically?}\label{subsec:measurement-quality}
The metric allows developers to be ranked by suitability for the position.
The qualities of this ranking are ultimately determined by whether
or not the right candidates land at the top of it. This needs to be
verified with real world job candidates and the job opening descriptions
that they applied for


\section{Related Work}

Jennifer Marlow and Laura Dabbish\cite{md:2013} from Carnegie Mellon
University have conducted an extensive study with about
200 participants, and found out that employers value
open source coder profiles greatly (page 4). Further in the document,
the two authors prove that developers are conscious of the fact
that their source code is public and thus try to deliver it with
great care. This hints at the chance being high that the code used for measurements
is written with care and thoughts have gone into the design of it.
Altogether their research has proven, that both public sourcecode repositories
are relevant for making recruitment decisions and the developers
are aware of this fact and try to deliver their best.
\newline

Philipp Giese has proven in his master thesis\cite{pg:2014} that
static code analysis is suitable for identifying personal skillsets.
He built a framework called \textit{Analyzr}\footnote{\url{https://github.com/frontendphil/analyzr}}
that analyzes the complete history from a git or SVN source code repository.
Based on the code, different metrics are calculated, from which statements about
code complexity and refactoring needs can be derived. It is also possible
to make statements about developer capabilites and areas of expertise.
Thus, given a little context about the analyzed source code, it would possible
to say that \glqq Developer X is mainly a backend developer and the single maintainer of module Y\grqq.
\newline

Philipp's development findings are supported by that of
Aftab Iqbal and Michael Hausenblas, who have analyzed different
source code repositories for common developers\cite{ih:2012}.
They have done so using relatively little local code analysis and relied
on on-the-fly analysis from retrieved RDF data. Their work
has made it possible to find out whether different open source projects shared developers
and contained useful information as to possible methods of retrieving authors
and eleminating duplicates caused by different e-mail addresses, for example.