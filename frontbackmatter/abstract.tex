%*******************************************************
% Abstract
%*******************************************************
%\renewcommand{\abstractname}{Abstract}
\pdfbookmark[1]{Abstract}{Abstract}
\begingroup
\let\clearpage\relax
\let\cleardoublepage\relax
\let\cleardoublepage\relax

\chapter*{Abstract}
The volume of data produced on the Internet every day is growing rapidly. Especially social networks grow vastly and get filled with more and more content each second. The size of these networks has made them attractive as marketing platforms, but also for recruiting.

Specifically in the technology industry, talented colleagues are scarce and heavily sought after. While it is possible to go through social networks manually to hire top talent, it is not advisable, for a lot of data can not be covered due to lack of manpower. To harness greater processing power spanning over several social networks, a metric has been put together for comparing developers and matching them on to job opening requirements. For verifying it, a prototypical platform where recruiters can find developers by stating programming language requirements has been implemented.

The results of a comparison of the recommendations for a position with the opinions of the candidates about their suitability for it resulted in about 80\% accuracy. The metric is based on data from analyzed source code repositories and GitHub.


\vfill

\pdfbookmark[1]{Zusammenfassung}{Zusammenfassung}
\chapter*{Zusammenfassung}
Mit zunehmender Verbreitung von sozialen Netzwerken w\"achst deren Wichtigkeit als Vertriebskanal - aber auch f\"ur die Pr\"asentation des Unternehmens. Dennoch funktioniert das Anwerben von Mitarbeiten immernoch auf dem klassischen Wege von Anschreiben samt Lebenslauf.
Beim Einstellen von besonders stark nachgefragten Programmierern stehen Firmen sich mit zus\"atzlichen Testprojekten und Pr\"ufungsfragen oft selbst im Weg. In dieser Arbeit wird eine auf Quellcode-Analyse basierende Metrik beschrieben, anhand derer man Kandidaten f\"ur eine Entwicklerstelle leicht anhand der Anforderungen f\"ur diese Stelle identifizieren kann. Mittels einer Plattform, auf der sich Entwickler registrieren k\"onnen, und auf der anhand von Programmierfertigkeitskriterien gesucht werden kann, wurden Daten f\"ur die Verifizierung des Verfahrens gesammelt. In einer Reihe von Interviews konnte eine \"Ubereinstimmung von ca. 80\% von Selbsteinsch\"atzung und algorithmisch bestimmter Eignung festgestellt werden.


\endgroup

\vfill