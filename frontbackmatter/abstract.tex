%*******************************************************
% Abstract
%*******************************************************
%\renewcommand{\abstractname}{Abstract}
\pdfbookmark[1]{Abstract}{Abstract}
\begingroup
\let\clearpage\relax
\let\cleardoublepage\relax
\let\cleardoublepage\relax

\chapter*{Abstract}
The working title of the project that preceded this thesis was \glqq Noise To Opportunity\grqq.
It was about discovering sales opportunities in social media posts from various
sources, matching them to a product and handing the post over to a company representative
responsible for this area of sales. The representative then had the option to
assess the real buying interest behind the posting and follow the opportunity or dismiss it as irrelevant.

The underlying principle of matching products to postings about products is very generic
and this thesis is about porting it from a business-to-business-perspective to
a business-to-employee-perspective in the context of recruiting.
In this case vacant positions and products as well
as customers and employees are the same thing. The company is trying to get rid of both
products and vacant positions, trying to match them to their specific target groups.

Technical recruiting is elaborate because it requires a lot of tests or example projects.
By assessing candidates through the source code they have published on the internet,
it is possible to make some of the candidate pre-selection ahead of the interview.
To assess this idea, an application to match GitHub users to job postings has been created.
Along with an outlook on further possibilities of this recruiting method, this thesis will
also explain the theory of matching interests to candidates in general.
as the theory of matching data behind it.


\vfill

\pdfbookmark[1]{Zusammenfassung}{Zusammenfassung}
\chapter*{Zusammenfassung}
Kurze Zusammenfassung des Inhaltes in deutscher Sprache\dots


\endgroup

\vfill